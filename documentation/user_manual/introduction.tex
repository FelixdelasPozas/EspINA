\section{Introduction}

\espina{} is a tool for segmenting, editing and analyzing neuroscientific images acquired using microscopy.
The program is been developed by the CesViMa (Centro de Supercomputación y Visualización de Madrid)
for the Cajal Blue Brain Project.

\section{Supported Files}
\espina{} con read images in the following formats:
\begin{itemize}
  \item TIFF (Tagged Image File Format)
  \item MHD/MHA (MetaImage)\\
% note on mhd, mha formats
\begin{bclogo}[couleur = yellow!33, logo=\bcattention]
{Note} In MetaImage files using the character \% in the name of the file are unsupported.
\end{bclogo}
  \item SEGMHA (former \espina{} format, provided for compatibility)
  \item SEG (\espina{} format)
\end{itemize}
TIFF and MHD/MHA formats contains channels, while SEGMHA format contains both the channel and the
segmentation images, and SEG formats contain segmentation images and additional data. SEG 
files containt a reference to the channel image used for segmentation. Segmentation images
produced by \espina{} con only be saved in SEG format.
